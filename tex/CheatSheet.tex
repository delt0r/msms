\documentclass{article}
\usepackage{amssymb}
\usepackage{amsfonts}
\usepackage{amsmath}
\usepackage{amsthm}
\usepackage{url}
\usepackage{graphicx}
\usepackage{eqlist}
\usepackage{natbib}
\usepackage{fancyvrb}
\usepackage{color}
\usepackage{fullpage}

\usepackage[table]{xcolor}

%\usepackage[perpage,symbol*]{footmisc}

%\DefineFNsymbols*{daggers}{\dagger\ddagger{\dagger\suck\dagger}{\ddagger\ddagger}*{**}{***}}
%\setfnsymbol{daggers}

 
\begin{document}
\title{$msms$ Cheat Sheet} 
\maketitle
This cheat sheet is for the program outlined in \cite{ewing_2010}.
\begin{eqlist*}{\topsep5em}
\item[{\tt -help}] Print out options documentation. This 
is often more up to date than this document. If you are unsure of a option, this
provides invaluable ``live'' documentation.
\item[{\tt -ms} nsamples nrep \quad] The total number of samples and number of
replicates. The {\tt -ms} can be omitted if these are the first two arguments.
\item[{\tt -N} $N_e$] Set $N_e$, note that event times are in
discrete generation times in units of $4N_e$. Not required if there is no
selection. 
\item[{\tt -t} $\theta$] Set the value of $\theta=4N_e\mu$
\item[{\tt -s} $s$] Condition on the number of segregating sites. Just a little
slower than using {\tt -t} and uses more memory.
\item[{\tt -T}] Output gene trees.
\item[{\tt -L}] Output tree length statistics. 
\item[{\tt -r} $\rho$ {[nsites]} ] Set recombination rate $\rho=4N_er$ where
$r$ is the recombination rate between the ends of a unit length sequence. If nsites
are omitted then an infinite sites recombination model is used.   
\item[{\tt -G} $\alpha$] Set growth parameter of all populations to $\alpha$. 
\longitem[{\tt -I} npop n1 n2 \ldots {[$4N_em$]}] Set up a structured population
model. The sample configuration must add up to the same total number of samples
as specified by {\tt -ms}.
\item[{\tt -n} $i$ $x$] Set the size of subpopulation $i$ to $xN_e$. 
\item[{\tt -g} $i$ $\alpha_i$] Set the growth rate of subpopulation $i$ to
$\alpha_i$.
\item[{\tt -m} $i$ $j$ $M_{ij}$] Set the $(i,j)$ element of the migration
matrix to $M_{ij}$.
\item[{\tt -ma} $M_{11}$ \ldots] Set the entire migration matrix.
\item[{\tt -eM} $t$ $x$] Set all elements of the migration matrix at time $t$
to $x/(\textrm{npop}-1)$
\item[{\tt -es} $t$ $i$ $p$] Split subpopulation $i$ into subpopulation $i$ and
npop$+1$ pastward. Each lineage currently in subpopulation $i$ is retained with
probability $p$, otherwise it is moved to the new population. The migration
rates to the new subpopulation are zero and its population size is set to $N_e$. 
\item[{\tt -ej} $t$ $i$ $j$] Join subpopulation $i$ to subpopulation $j$. All
migration matrix entries with subpopulation $i$ are set to zero. The population
size of $i$ is also set to zero. With selection this population is ignored
pastward from this time.\\
\textcolor{red}{WARNING: This switch behaves differently from $ms$ in the strict
definition. We consider that most people expect that {\tt -ej} is modeling a 
split in forward time and hence the deme $i$ is turned off pastward. }
\item[{\tt -e[X]} $t$ \ldots] Set some parameter pastward from time $t$. Here
{\tt [X]} can be any of {\tt G g n m ma} and the meaning is defined as for the
normal command, for example {\tt -en} $t$ $i$ $x$ sets the population size of
deme $i$ to $xN_e$ pastward from time $t$.
\item[{-l} $n$ $a_1$ $a_1^{'}$ \ldots $a_n$ $a_n^{'}$] Set the neutral loci
starting and stopping positions for $n$ loci. Note that must be $a_i^{'}<a_{i+1}$
for all $i$ and that there must be $2n$ values. All parameters assume a
sequence length of 1. This other parameter needs to be scaled accordingly.   
\item[{\tt -SAA} $\alpha_{AA}$] Set the selection strength of the homozygote in
units of $2N_es$.
\item[{\tt -SAa} $\alpha_{Aa}$] Set the selection strength of the heterozygote
in units of $2N_es$.
\item[{\tt -Smu} $4N_e\mu^{'}$] Set the forward mutation rate for the selected
allele. That is the mutation from the wild type $a$ to
derived type $A$.
\item[{\tt -Snu} $4N_e\nu^{'}$] Set the backward mutation rate for the selected
allele. That is the mutation from the selected type
$A$ to the wild type $a$.
\item[{\tt -Sp} $x$] Set the position $x$ in the sequence of the selected allele.
\longitem[{\tt -Sc} $t$ $i$ $\alpha_{AA}$ $\alpha_{Aa}$ $\alpha_{aa}$] Set the
selection strength in deme $i$ to the specified values pastward from time
$t$.$\alpha$ is in units of $2N_es$
\item[{\tt -SF} $t$ \\{\tt -SF} $t$ $f$ \\{\tt -SF} $t$ $i$ $f$] 
Set the selection simulation stopping condition to
fixation at time $t$ pastward from sampling time. $t$ is time into the past,
$i$ is the deme and $f$ is the frequency. The first case assumes fixation
across all populations, the second case assumes frequency is across all
populations. Selection is not used forward in time from this point. It is up to
the user to ensure that the parameters permit the model to always go to
fixation,  otherwise it will keep simulating till it runs out of memory. \\
\textcolor{red}{ Note the demographic model must be time invariant for this option to work properly.}
\longitem[{\tt -SI} $t$ npop $x_1$ $x_2$ \ldots] Set the start of selection to
time $t$ {\em forward} in time from this point. The initial frequencies of the
beneficial allele are $x_1,x_2,\ldots$. Note that this option is not compatible with
{\tt -SF}.
\item[{\tt -Smark}] Include the selected locus in the mutation output. 
\longitem[{\tt -oTPi} $w$ $s$ {\tt [onlySummary]}] Output windowed 
$\theta$ estimates (both Wattersons and $\pi$ based estimators) and Tajima's D
with window size $w$ and step size $s$. If {\tt onlySummary}, then only the
averages of all replicates are output. The output format is a table formatted as follows:
The first column is the bin position. The second column is the Watterson's
$\theta$ estimator. The third column is the $\pi$\footnote{average pairwise
difference} estimator and the last column is Tajima's D. The summary also
contains the standard deviations for the previous data column. Thus, column 3 is
the standard deviation of the Wattersons $\theta$ estimators. 
\item[{\tt -oOC}] Output the number of origins of the beneficial allele in the
sample. A count of 0 or 1 means a hard sweep if conditioned on fixation. 
\longitem[{-tt -oAFS} {\tt [jAFS]} {\tt [onlySummary]} ] Output allele  
frequency spectra. If the {\tt jAFS} option is specified, all pairwise deme
joint frequency spectra are output.
\item[{\tt -oTrace}] Print the frequency trajectory of the forward
simulations. The first column is the time in $4N_e$ generations pastward from
present. Then each column is the relative frequency of the beneficial allele in
each deme. This format is the same as required when specifying a trajectory. 
\item[{\tt -Strace} $filename$] Rather than simulate the forward trajectory,
specify the trajectory in a text file. The format of the {\tt -oTrace} option
is valid for input. Note that you must include ``unused'' demes produced with
{\tt -ej} or {\tt -es} options even if the frequency is zero. Also it is not
required to specify every generation. Just a time and frequencies in a
decreasing order. $msms$ will use linear interpolation for generations between
specified time points. 
\item[{\tt -threads} $n$] Specify the number of threads to use. This permit
very easy use of multicore machines. The number of threads should be only as much as
you have cores available. This will increase memory by the same factor as
threads, so 2 threads will use twice as much memory as one. Also this is not
effective if each simulation is very fast, as the cores spend most of their time
waiting to output data. 
\item[{\tt -seed} $v$] Set the seed. The seed is very different from $ms$
since $msms$ uses quite a different random number generator. This is a
64 bit number that can be specified either in hex with a {\tt 0x} prefix or
normal decimal. $msms$ goes to some effort to randomize the seed value so you
you don't need to set seed values on cluster environment. When using the
{\tt -threads} the seed for ``iteration $n$'' will be the same and hence give
the same result. However the order of reported results will generally be
different. 
\end{eqlist*}

\bibliographystyle{applicationNote/natbib} 
\bibliography{applicationNote/Bio} 


\end{document}











